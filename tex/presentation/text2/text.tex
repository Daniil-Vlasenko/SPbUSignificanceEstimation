\documentclass[ucs, notheorems, handout, 10pt]{beamer}

\usetheme[numbers,totalnumbers,compress, nologo]{Statmod}
%\usetheme{Madrid}
\usefonttheme[onlymath]{serif}
\setbeamertemplate{navigation symbols}{}

\mode<handout> {
	\usepackage{pgfpages}
	\setbeameroption{show notes}
	\pgfpagesuselayout{2 on 1}[a4paper, border shrink=5mm]
	\setbeamercolor{note page}{bg=white}
	\setbeamercolor{note title}{bg=gray!10}
	\setbeamercolor{note date}{fg=gray!10}
}

\usepackage[utf8x]{inputenc}
\usepackage[T2A]{fontenc}
\usepackage[russian]{babel}
\usepackage{tikz}
\usepackage{ragged2e}
\usepackage{amsmath,amssymb,amsthm,amscd,amsfonts, mathabx}
\setbeamertemplate{caption}[numbered]

\newtheorem{theorem}{Теорема}
\newtheorem{defenition}{Определение}

\title[Оценивание значимости выравнивания]{Задачи оценивания значимости выравнивания при помощи скрытых марковских моделей}

\author[Власенко Д.В.]{Власенко Даниил Владимирович, гр.19.Б04-мм}

\institute[Санкт-Петербургский Государственный Университет]{%
	\small
	Научный руководитель: к.ф.-м.н. Коробейников А.И.\\ \vspace{0.5cm}
	Санкт-Петербургский государственный университет\\
	Прикладная математика и информатика\\
	Вычислительная стохастика и статистические модели\\
	\vspace{1.25cm}
	Отчет по производственной практике}

\date[Зачет]{Санкт-Петербург, 2022}

\subject{Talks}


\begin{document}
	
	\begin{frame}[plain]
		\titlepage
		
		\note{Научный руководитель  к.ф.-м.н., Коробейников А.И.,\\
			кафедра статистического моделирования}
	\end{frame}
	
	\begin{frame}{Введение}
		Пусть дан алфавит символов $\Sigma$. 
		\begin{defenition}Последовательностью длины $L$ над алфавитом $\Sigma$ будем называть такой элемент $X$, что $X \in \Sigma^{L}$. Последовательностью $X$ над алфавитом $\Sigma$ будем называть такой $X$, что $X \in \bigcup_{L=0}^{L=\infty}\Sigma^{L}$.
		\end{defenition}
		
		Сходство последовательностей может отражать взаимосвязи объектов, которые они описывают. Например, такие как:

		\begin{center}
			\begin{itemize}
				\item функциональные,
				\item структурные,
				\item эволюционные.
			\end{itemize}
		\end{center}		 
		
		\note{
			Пусть дан алфавит символов $\Sigma$. 
			\begin{defenition}Последовательностью длины $L$ над алфавитом $\Sigma$ будем называть такой элемент $X$, что $X \in \Sigma^{L}$. Последовательностью $X$ над алфавитом $\Sigma$ будем называть такой $X$, что $X \in \bigcup_{L=0}^{L=\infty}\Sigma^{L}$.
			\end{defenition}
			
			Сходство последовательностей может отражать функциональные, структурные или эволюционные взаимосвязи объектов, которые описывают эти последовательности. Таким образом умение находить взаимосвязи в строках может быть приложимо в задаче определения степени родства биологических организмов путем сравнения их ДНК или РНК, нуклеотидных последовательностей, задаче анализа свойств белков, аминокислотных последовательностей, задаче распознавания речи человека или письменного языка и многих других приложениях. 
		}
	\end{frame}

	\setcounter{framenumber}{1}
	\begin{frame}{Введение}
		\begin{defenition}
			Выравниванием $N$ последовательностей называется отображение $Q: \bigtimes_{i=1}^{N}(\bigcup_{L_i=0}^{L_i=\infty} \Sigma^{L_i}) \rightarrow \bigtimes_{i=1}^{N}(\Sigma^{\max_{L \in L_i}(L)})$, такое что:
			\begin{enumerate}
				\item Возможны вставки символа --- в последовательностях.
				\item Вставка --- на одинаковых позициях во всех последовательностях запрещена.
				\item Порядок изначальных символов внутри последовательностей сохраняется.
			\end{enumerate}
		\end{defenition}
		
		Элементы из области значения $Q$ также называются выравниваниями.
		
		\note{
			\begin{defenition}
				Выравниванием $N$ последовательностей называется отображение $Q: \bigtimes_{i=1}^{N}(\bigcup_{L_i=0}^{L_i=\infty} \Sigma^{L_i}) \rightarrow \bigtimes_{i=1}^{N}(\Sigma^{\max_{L \in L_i}(L)})$, такое что:
				\begin{enumerate}
					\item Возможны вставки символа --- в последовательностях.
					\item Вставка --- на одинаковых позициях во всех последовательностях запрещена.
					\item Порядок изначальных символов внутри последовательностей сохраняется.
				\end{enumerate}
			\end{defenition}
			
			Элементы из области значения $Q$ также называются выравниваниями.
		}
	\end{frame}

	\begin{frame}{Введение}
		\begin{figure}
			\begin{tabular}{cccccccc}
				A&C&E&A&A&F&A&E\\
				C&E&A&F&D&C&E&\\
			\end{tabular}
		\end{figure}
		\begin{figure}
			\begin{tabular}{ccccccccc}
				A&C&E&A&A&F&A&—&E\\
				—&C&E&A&—&F&D&C&E\\
			\end{tabular}
			\caption{Последовательности до и после выравнивания.} \label{fg:1}
		\end{figure}
	
		Примем множество $\bigtimes_{i=1}^{N}(\bigcup_{L_i=0}^{L_i=\infty} \Sigma^{L_i})$ за пространство элементарных исходов $\Omega$. Область значений выравнивания $Q$ обозначим как $\overline \Omega$.
		
		\begin{defenition}				
			Оценкой выравнивания называется случайная величина $s:\overline \Omega \rightarrow \mathbb{R}$.
		\end{defenition}			
		
		\note{
			Примем множество $\bigtimes_{i=1}^{N}(\bigcup_{L_i=0}^{L_i=\infty} \Sigma^{L_i})$ за пространство элементарных исходов $\Omega$. Область значений выравнивания $Q$ обозначим как $\overline \Omega$.
			
			\begin{defenition}				
				Оценкой выравнивания называется случайная величина $s:\overline \Omega \rightarrow \mathbb{R}$.
			\end{defenition}	
		
			Способом вычисления оценки выравнивания $s$ может быть, например, увеличение оценки на 1 при совпадении символов, стоящих на одинаковых позициях в последовательностях, и уменьшение на $\frac{1}{2}$ при несовпадении. Тогда оценка $s$ приведенного на слайде выравнивания будет равна 3.
			
			Определить оценку выравнивания можно разными способами, но смысл будет иметь такое определение, чтобы оценка была мерой того, насколько сильно строки выравнивания похожи друга на друга.
		}
	\end{frame}

	\setcounter{framenumber}{2}
	\begin{frame}{Введение}
		Пусть даны последовательности $\{X_i\}_{i=1}^{N}$ и задана оценка выравнивания $s$. Тогда задача оценки сходства последовательностей $\{X_i\}_{i=1}^{N}$ сводится к решению оптимизационной задачи:
			
		\begin{equation*}
			\max_{\overline{\omega} \in \overline{\Omega} : Q(\{X_i\}_{i=1}^{N}) = \overline{\omega}}s(\overline{\omega}).
		\end{equation*}	
		
		\note{
			Пусть даны последовательности $\{X_i\}_{i=1}^{N}$ и задана оценка выравнивания $s$. Тогда задача оценки сходства последовательностей $\{X_i\}_{i=1}^{N}$ сводится к решению оптимизационной задачи:
			
			\begin{equation*}
				\max_{\overline{\omega} \in \overline{\Omega} : Q(\{X_i\}_{i=1}^{N}) = \overline{\omega}}s(\overline{\omega}).
			\end{equation*}				
		}
	\end{frame}
	
	\begin{frame}{Введение}					
		Предположим, что даны строка $X$ и $\omega \in \overline{\Omega}$ из $N$ строк. 
		
		\begin{defenition}
			Выравниванием последовательности $X$ к выравниванию $w$ называется отображение $\boldsymbol{Q}: (X, \overline{\Omega}) \rightarrow \bigtimes_{i=1}^{N+1}(\Sigma^{\max_{L \in L_i}(L)})$, такое что:
			\begin{enumerate}
				\item Возможны вставки символа --- в последовательностях.
				\item Вставка --- на одинаковых позициях во всех последовательностях запрещена.
				\item Порядок изначальных символов внутри последовательностей сохраняется.
			\end{enumerate}
		\end{defenition}
		
		Примем множество $(\Omega, \overline \Omega)$ за пространство элементарных исходов $\boldsymbol{\Omega}$. Область значений выравнивания $\boldsymbol Q$ обозначим как $\overline{\boldsymbol{\Omega}}$.
		
		\begin{defenition}				
			Оценкой выравнивания последовательности $X$ к выравниванию $w$ называется случайная величина $\boldsymbol s:\overline{\boldsymbol{\Omega}} \rightarrow \mathbb{R}$.
		\end{defenition}				
		
		\note{							
			Предположим, что даны строка $X$ и $\omega \in \overline{\Omega}$ из $N$ строк. 
			
			\begin{defenition}
				Выравниванием последовательности $X$ к выравниванию $w$ называется отображение $\boldsymbol{Q}: (X, \overline{\Omega}) \rightarrow \bigtimes_{i=1}^{N+1}(\Sigma^{\max_{L \in L_i}(L)})$, такое что:
				\begin{enumerate}
					\item Возможны вставки символа --- в последовательностях.
					\item Вставка --- на одинаковых позициях во всех последовательностях запрещена.
					\item Порядок изначальных символов внутри последовательностей сохраняется.
				\end{enumerate}
			\end{defenition}
		
			Примем множество $(\Omega, \overline \Omega)$ за пространство элементарных исходов $\boldsymbol{\Omega}$. Область значений выравнивания $\boldsymbol Q$ обозначим как $\overline{\boldsymbol{\Omega}}$.
			
			\begin{defenition}				
				Оценкой выравнивания последовательности $X$ к выравниванию $w$ называется случайная величина $\boldsymbol s:\overline{\boldsymbol{\Omega}} \rightarrow \mathbb{R}$.
			\end{defenition}
		}
	\end{frame}

	\begin{frame}{Введение}					
		Пусть даны строка $X$, выравнивание $\overline{\omega} \in \overline{\Omega}$ и задана оценка выравнивания $\boldsymbol{s}$. Тогда задача оценки сходства последовательности $X$ и множества, описываемого $\overline{\omega}$, сводится к решению оптимизационной задачи:
		
		\begin{equation*}
			\max_{\overline{\boldsymbol{\omega}} \in \overline{\boldsymbol{\Omega}} : \boldsymbol{Q}(X, \overline{\omega}) = \overline{\boldsymbol{\omega}}}s(\overline{\boldsymbol{\omega}}).
		\end{equation*}	
		
		\note{	
			Пусть даны строка $X$, выравнивание $\overline{\omega} \in \overline{\Omega}$ и задана оценка выравнивания $\boldsymbol{s}$. Тогда задача оценки сходства последовательности $X$ и множества, описываемого $\overline{\omega}$, сводится к решению оптимизационной задачи:
			
			\begin{equation*}
				\max_{\overline{\boldsymbol{\omega}} \in \overline{\boldsymbol{\Omega}} : \boldsymbol{Q}(X, \overline{\omega}) = \overline{\boldsymbol{\omega}}}s(\overline{\boldsymbol{\omega}}).
			\end{equation*}									
		}
	\end{frame}

	\begin{frame}{Введение}					
		
		
		\note{							
			Так как последовательности описывают некоторые объекты, то введенную на предыдущем слайде оценку можно интерпретировать как оценку того, насколько сильно новая последовательность $X$ близка к множеству последовательности, на которых было построено выравнивание $\omega$.
			
			Встает вопрос того, как интерпретировать величину этой оценки:
			\begin{itemize}
				\item достаточно ли высокая эта оценка, чтобы считать объект, описываемый последовательностью $X$, родственным к объектам, описываемым выравниванием $w$, или шум, т.е. случайная последовательность, мог получить такую оценки.
				\item достаточно ли низкая эта оценка, чтобы считать объект описываемый последовательностью $X$, не родственным к объектам, описываемым выравниванием $w$, или сигнал, т.е. последовательность, описывающая взаимосвязанный с множеством объект, мог получить такую оценку. 
			\end{itemize}
		}
	\end{frame}
	
\end{document}
