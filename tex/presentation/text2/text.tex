\documentclass[ucs, notheorems, handout, 10pt]{beamer}

\usetheme[numbers,totalnumbers,compress, nologo]{Statmod}
%\usetheme{Madrid}
\usefonttheme[onlymath]{serif}
\setbeamertemplate{navigation symbols}{}

\mode<handout> {
	\usepackage{pgfpages}
	\setbeameroption{show notes}
	\pgfpagesuselayout{2 on 1}[a4paper, border shrink=5mm]
	\setbeamercolor{note page}{bg=white}
	\setbeamercolor{note title}{bg=gray!10}
	\setbeamercolor{note date}{fg=gray!10}
}

\usepackage[utf8x]{inputenc}
\usepackage[T2A]{fontenc}
\usepackage[russian]{babel}
\usepackage{tikz}
\usepackage{ragged2e}
\usepackage{amsmath,amssymb,amsthm,amscd,amsfonts, mathabx}
\setbeamertemplate{caption}[numbered]

\newtheorem{theorem}{Теорема}
\newtheorem{defenition}{Определение}

\title[Оценивание значимости выравнивания]{Задачи оценивания значимости выравнивания при помощи скрытых марковских моделей}

\author[Власенко Д.В.]{Власенко Даниил Владимирович, гр.19.Б04-мм}

\institute[Санкт-Петербургский Государственный Университет]{%
	\small
	Научный руководитель: к.ф.-м.н. Коробейников А.И.\\ \vspace{0.5cm}
	Санкт-Петербургский государственный университет\\
	Прикладная математика и информатика\\
	Вычислительная стохастика и статистические модели\\
	\vspace{1.25cm}
	Отчет по производственной практике}

\date[Зачет]{Санкт-Петербург, 2022}

\subject{Talks}


\begin{document}
	
	\begin{frame}[plain]
		\titlepage
		
		\note{Научный руководитель  к.ф.-м.н., Коробейников А.И.,\\
			кафедра статистического моделирования}
	\end{frame}
	
	\begin{frame}{Введение}
		Пусть дан алфавит символов $\Sigma$. 
		\begin{defenition}Последовательностью длины $L$ над алфавитом $\Sigma$ будем называть такой элемент $X$, что $X \in \Sigma^{L}$. Последовательностью $X$ над алфавитом $\Sigma$ будем называть такой $X$, что $X \in \bigcup_{L=0}^{L=\infty}\Sigma^{L}$.
		\end{defenition}
		
		\begin{defenition}
			Парное выравниванием последовательностей называется отображение $Q: (\bigcup_{L_1=0}^{L_1=\infty} \Sigma^{L_1} \bigtimes \bigcup_{L_2=0}^{L_2=\infty}\Sigma^{L_2}) \rightarrow (\Sigma^{\max(L_1, L_2)} \bigtimes $ $ \Sigma^{\max(L_1, L_2)})$, такое что:
			\begin{enumerate}
				\item Возможны вставки символа --- в последовательностях.
				\item Вставка --- на одинаковых позициях в обоих последовательностях запрещена.
				\item Порядок изначальных символов внутри последовательностей сохраняется.
			\end{enumerate}
		\end{defenition}		
		
		\note{
			Пусть дан алфавит символов $\Sigma$. 
			\begin{defenition}Последовательностью длины $L$ над алфавитом $\Sigma$ будем называть такой элемент $X$, что $X \in \Sigma^{L}$. Последовательностью $X$ над алфавитом $\Sigma$ будем называть такой $X$, что $X \in \bigcup_{L=0}^{L=\infty}\Sigma^{L}$.
			\end{defenition}
			
			\begin{defenition}
				Парное выравниванием последовательностей называется отображение $Q: (\bigcup_{L_1=0}^{L_1=\infty} \Sigma^{L_1} \bigtimes \bigcup_{L_2=0}^{L_2=\infty}\Sigma^{L_2}) \rightarrow (\Sigma^{\max(L_1, L_2)} \bigtimes $ $ \Sigma^{\max(L_1, L_2)})$, такое что:
				\begin{enumerate}
					\item Возможны вставки символа --- в последовательностях.
					\item Вставка --- на одинаковых позициях в обоих последовательностях запрещена.
					\item Порядок изначальных символов внутри последовательностей сохраняется.
				\end{enumerate}
			\end{defenition}
		}
	\end{frame}

	\begin{frame}{Введение}
		\begin{figure}
			\begin{tabular}{cccccccc}
				A&C&E&A&A&F&A&E\\
				C&E&A&F&D&C&E&\\
			\end{tabular}
		\end{figure}
		\begin{figure}
			\begin{tabular}{ccccccccc}
				A&C&E&A&A&F&A&—&E\\
				—&C&E&A&—&F&D&C&E\\
			\end{tabular}
			\caption{Последовательности до и после парного выравнивания.} \label{fg:1}
		\end{figure}
	
		Примем множество $\bigcup_{L_1=0}^{L_1=\infty} \Sigma^{L_1} \bigtimes \bigcup_{L_2=0}^{L_2=\infty}\Sigma^{L_2}$ за пространство элементарных исходов $\Omega$. Область значений выравнивания $Q$ обозначим как $\overline \Omega$.
		
		\begin{defenition}			
			Оценкой парного выравнивания называется случайная величина $s:\overline \Omega \rightarrow \mathbb{R}$.
		\end{defenition}			
		
		\note{
			Примем множество $\bigcup_{L_1=0}^{L_1=\infty} \Sigma^{L_1} \bigtimes \bigcup_{L_2=0}^{L_2=\infty}\Sigma^{L_2}$ за пространство элементарных исходов $\Omega$. Область значений выравнивания $Q$ обозначим как $\overline \Omega$.
			
			\begin{defenition}
				
				Оценкой парного выравнивания называется случайная величина $s:\overline \Omega \rightarrow \mathbb{R}$.
			\end{defenition}		
		}
	\end{frame}
	
\end{document}
